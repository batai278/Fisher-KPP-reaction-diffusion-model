\begin{lstlisting}
class CNModel(object):
      '''
  1) dt - time discretization step (mandatory)
  2) dx, dy - coordinate discretization on a plane for Ox and Oy consequently (mandatory)
  3) lambd - factor in Crank-Nicholson model (mandatory)
  4) T,M,N - number of nodes for time, Ox, Oy (int expected)
  
  init_cond could be eiter callable or numerical; if float(int) given casting to constant function occurs
      '''
  
      def __init__(self, lambd, dt, dx, dy, T, M, N, sigma=1., alpha=1., beta=1., m_0=1., init_cond=None):
          self.lambd = lambd
          self.dt = float(dt)
          self.dx = float(dx)
          self.dy = float(dy)
          if (type(T) is not int or type(M) is not int or type(N) is not int):
              raise ValueError("Nodes number (T,M,N): int expected")
          self.T = T  # self.T = int(T/dt)+1
          self.M = M  # self.M = int(M/dx)+1
          self.N = N  # self.N = int(N/dy)+1
          self.domain = (self.dt * self.T, self.dx * (self.M - 1), self.dy * (self.N - 1))
          print "Model on (0,%s) time interval, [0,%s]x[0,%s] plane created" % self.domain
          self.alpha = alpha
          self.beta = beta
          self.m_0 = m_0
          if callable(init_cond):
              self.init_cond = init_cond
          else:
              self.init_cond = cast_init_to_fun(init_cond)
          self.sigma = sigma
  
      def _get_footprint(self):
          return {'up': -self.lambd * self.S_x,
                  'down': -self.lambd * self.S_x,
                  'left': -self.lambd * self.S_y,
                  'right': -self.lambd * self.S_y,
                  'center': 1. - self.lambd * self.dt * self.m_0 * self.alpha + 2. * self.lambd * self.S
                  }
  
      def prepare(self):
          self.n_0 = float(self.m_0) / self.beta
          self.S_x = (self.sigma * self.dt) / (self.dx)**2
          self.S_y = (self.sigma * self.dt) / (self.dy)**2
          self.S = self.S_x + self.S_y
          self.footprint = self._get_footprint()
          self.initial_state = self.generate_initial_state()
  
      def generate_initial_state(self):
          init_grid = []  # np.ones((self.mod.M*self.mod.N), dtype=float)
          for i in range(self.M):
              for j in range(self.N):
                  init_grid.append(self.init_cond([i * self.dx, j * self.dy]))
          return np.asarray(init_grid, dtype=float)
\end{lstlisting}
