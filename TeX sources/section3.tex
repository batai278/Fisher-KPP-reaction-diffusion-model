\section{Аналіз стійкості обраної різницевої схеми}
%\subsection{Попередні зауваження}
Для аналізу стійкості відповідної лінеаризованої різницевої схеми \eqref{eq:syst} будемо користуватися спектральним методом фон Неймана, детальний розгляд якого та
доведення відповідних теорем необхідності та достатності (в першу чергу для лінійного випадку) можна знайти у \cite{godun, richt}. 

Для початку визначимо згідно з \cite{godun} різницеву схему як \textit{стійку відносно збурень вхідних даних}, якщо для її розв’язку виконано:
\begin{equation}
\smash{\displaystyle\max_{j,k} |\un{0}{0}|} \le  C\smash{\displaystyle\max_{j,k} |u^0_{j,k}|},\  \forall n = 0, \dots, T,\ C = const,\ |C| < \infty
\end{equation}
для довільної обмеженої ф-ї $u^0_{j,k} = \psi_{j,k}$.

Слідуючи за \cite{richt} знайдемо коефіцієнт переходу (підсилення) $\xi$ для Фур’є-компоненти точного розв’язку різницевої схеми. Для нелінійних ДРЧП де-факто він залежить не лише від набору констант, що характеризують р-ня, і обраних кроків сітки, але й, власне кажучи, від самих розв’язків, що отримуються, тому можна лише сподіватись, що стійкість буде мати місце до деякого моменту часу $t_1$; крім того часто строгі аналітичні оцінки на коефіцієнт переходу є важкими і громіздкими, тому у даній роботі перевірку отриманої умови (в якій є явна залежність від $u^n_{j,k}$) було реалізовано у програмному коді і у випадку коли умова перестає виконуватись моделювання припинялось (в якості покращення методу, можна було б підганяти крок по часу, щоб задовільнити умову стійкості).

Розглянемо n-ту Фур’є-компоненту точного розв’язку схеми: $\un{0}{0} = \xi^ne^{imj\Delta x}e^{ilk\Delta y}$; підставимо її у \eqref{eq:syst} з лінеаризованим квадратичним членом і після скорочень, виражаючи $\xi$ отримаємо:
\begin{equation}
\xi = \frac{1 - (1-\lambda)\left[\Delta t\alpha m_0 + 2\psi\left(m\Delta x, l\Delta y\right)\right] - (1-2\lambda)\Delta t\alpha\beta\times u^n\left(m\Delta x, l\Delta y\right)}
{1 - \lambda\left[\Delta t m_0\alpha - 2\psi\left(m\Delta x, l\Delta y\right)\right] + 2\lambda\Delta t\alpha\beta\times u^n\left(m\Delta x, l\Delta y\right)} \label{eq:stable}
\end{equation}
де $\psi(m\Delta x, l\Delta y) = S_x\{1-\cos(m\Delta x)\} + S_y\{1-\cos(l\Delta y)\}$, $S_x = \frac{\sigma\Delta t}{\Delta x^2},\ S_y = \frac{\sigma\Delta t}{\Delta y^2}$ --- відповідні числа дифузії.

Відомо, що необхідною умовою стійкості є
\begin{equation}
\smash{\displaystyle\max_{m,l} |\xi|} < 1
\end{equation}
отже саме її і будемо використовувати у програмній реалізації.