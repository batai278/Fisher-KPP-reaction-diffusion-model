\section{Попередній огляд}
\subsection{Постановка задачі}

У даній роботі запропоновано числовий розв’язок р-ня \eqref{eq:fisher}, що моделює процес розмноження бактерій, який
здійснюється шляхом їх ділення.
\begin{equation}
\frac{\partial u}{\partial t} = \alpha (m_{0} -\beta u)u + \sigma\Delta u; \label{eq:fisher}
\end{equation}
Надалі будемо вважати, що процес відбувається в прямокутній  області $\Omega \in \mathbb{R}^{2} $  $(0 \le x \le a, \\
0 \le y \le b)$. Змінна $u$ позначає концентрацію мікроорганізмів у визначеній області.
Порогове значення концентрації $u_{max}=\frac{m_{0}}{\beta}$.

Початкові та граничні умови:
\begin{equation}
	\begin{cases} u\rvert_{t=0} = U(\vec{x}),\ &\forall \vec {x} \in \Omega \\
	u\rvert_\Gamma = u_{\Gamma},\ &\Gamma = \partial \Omega \end{cases} 
\end{equation}
де $U(\vec{x})$  --- гладка класу $\mathbf{C}^2(\Omega)$ функція; крім того для узгодженості початкових та граничних умов покладемо $U(\vec{x})\rvert_\Gamma = u_\Gamma$ i вважатимемо, що протягом усьго розвитку системи значення $u_\Gamma$ залишається постійним. \\
%makebox[5pt]{}
%\shadowbox{Примітка:} Конкретні експерименти проводилися з функцією , в якості якої було обрано двовимірну гаусіану.
\subsection{Розгляд рівняння}
Рівняння \eqref{eq:fisher} є узагальненням для двовимірного простору відомого р-ня Фішера (або частковим випадком з квадратичною нелінійністю р-ня Колмогорова-Петровського-Піскунова). В контексті популяційної динаміки одновимірний безрозмірний варіант цього р-ня був розглянутий  Р. Фішером у 1937 р. у \cite{fisher} де \eqref{eq:fisher} було запропоновано для опису просторового росподілу домінантних алелей і був проведений аналіз його розв’язку у вигляді біжучих хвиль. Стислий але змістовний огляд узагальнення цього р-ня Колмогоровим, а також умови існування його розв’язку у вигляді біжучих хвиль для одновимірного випадку можна знайти в \cite{treft}.

Модель, що задається р-ням \eqref{eq:fisher}, відноситься до типу реакційно-діфузійних моделей. Такі моделі поєднують в собі опис просторово розподіленої хімічної (біологічної, тощо) реакції --- в нашому р-ні представлена логістичним нелінійним членом, --- з описом дифузії реагентів через субстрат --- член з лапласіаном у \eqref{eq:fisher}. Зазвичай такі моделі представляються системою ДРЧП, невідомі ф-ї в якій описують концентрації (зазвичай, або ж деяку іншу характеристику) реагентів (видів, популяцій у застосуванні до біології), що взаємодіють між собою і розповсюджуються у просторі. У нашому випадку розглядається одновидова модель розмноження бактерій на площині з одночасною дифузією. Ділення бактерій описується логістичним членом (який є узагальненням звичайної експоненційної моделі), що введений у \eqref{eq:fisher} для врахування граничної спроможності середовища вміщати у себе популяцію. Без цього нелінійного члена \eqref{eq:fisher} являє собою звичайне параболічне р-ня дифузії, тож модель, як було вказано, поєднує в собі опис розмноження бактерій разом з їх одночасним перерозподілом у просторі.

У таблиці нижче наведена інтерпретація параметрів р-ня з точки зору побудови моделі та біологічного змісту останніх, подробиці можуть бути знайдені у \cite{bio}.
\vspace{10pt}
\begin{center}
	\begin{tabular}{|l|l|}
	\hline
	Parameter & Meaning \\
	\hline
	$\alpha$  & Coef. of linear dependency between $\mathbf{R}(u)$ and "nutrient" \\ \hline
	$\beta$ & Coef. between decrease of "nutrient" and increase of population \\ \hline
	$m_{0}$  & Constant of availability of "nutrient" \\ \hline
	$\sigma$ & Coef. of diffusion \\ \hline
	$n_0 = \frac{m_0}{\beta}$ & Limit carrying capacity \\ \hline
	\end{tabular}
\end{center}
\vspace{10pt}
Тут символом $\mathbf{R}(u)$ позначено логістичний член $\alpha m_0u - \alpha\beta u^2$, який називають "growth rate", або темп росту; він визначає характер "реакційного" вкладу у реакційно-дифузних моделях. Як вказано в \cite{bio} для подібних моделей, що розглядаються у біології, саме квадратичні нелінійні члени є найбільш характерними. Зауважимо, що знову згідно \cite{bio} коефіцієнт $ - \alpha\beta$ визначає (оскільки маємо одну популяцію) внутрішньовидову конкуренцію при $\alpha\times\beta > 0$. Для біологів саме цей випадок є найцікавішим для досліджень у даній моеделі. Якщо коеф-т при квадратичному члені є додатнім, то цей випадок описує свого роду ``коменсалізм'', коли взаємодія окремих особин дає позитивний ефект для розвитку обох. Тим не менш цей окремий випадок нами не розглядався детально, оскільки, як показав експеримент, запропонована у роботі схема дискретизації виявляється суттево нестійкою при $\alpha\times\beta < 0$ (зокрема при $\alpha < 0$, навіть для дуже малих значеннях останнього).