\section{Опис програмної реалізації}
\subsection{Об’єктна модель}
Програмна реалізація розв’язку р-ня \eqref{eq:fisher} була розроблена за допомогою мови Python, версія 2.7.6, та додаткових програмних пакетів для деяких специфічних задач лінійної алгебри, що виникають у процесі розв’язку, а саме: \textbf{scipy (0.17.0)} (для представлення розрідженої матриці системи у форматі DIAGONAL та використання відповідних солверів для розріджених систем) та \textbf{numpy (1.11.0)} для базових представлень масивів та операцій лінійної алгебри. Відповідні лістинги наведені у \textbf{Додатку І}.

Об’єктна модель реалізації складається з двох класів: \textbf{CNModel} та \textbf{CNSolver}.
Перший репрезентує модель задачі, тобто зберігає в якості своїх атрибутів параметри рівняння та схеми дискретизації, функцію, що задає початкові та граничні умови, а також інкапсулює реалізацію методів, що готують модель для розв’язання та є основними методами, які викликає клієнтський код солвера.
Нижче наведено параметри, що передаються в конструктор класу при створенні моделі:
\vspace{10pt}
\begin{center}
	\begin{tabular}{|l|c|}
	\hline
	Parameters & Description \\ \hline
	alpha, beta, m\_0, sigma & $\alpha, \beta, m_0, \sigma$ --- Параметри р-ня (\textbf{float}) \\ \hline
	dt, dx, dy & $\Delta t, \Delta x, \Delta y$ --- кроки по часу, $Ox$, $Oy$ відповідно (\textbf{float})  \\ \hline
	T, M, N & Кількість вузлів по часу, $Ox$, $Oy$ відповідно (\textbf{integer}) \\ \hline
	lambd & $\lambda$ --- ваговий коеф-т (\textbf{float, belongs to [0, 1]}) \\ \hline
	init\_cond & $U(\vec{x})$ --- ф-я початкових умов (\textbf{either callable or float}) \\ \hline
	\end{tabular}
\end{center}
\vspace{10pt}
Методи классу включають в себе (Listing 1, Додаток І): \textbf{\_get\_footprint}, що одноразово обчислює коеф-ти в системі рівнянь, які стоять в кожному рядку при відповідних членах, \textbf{get\_initial\_state}, що формує сітку початкових значень, \textbf{prepare}, що є основним методом, який викликає клас \textbf{Solver}, і який готує модель до запуску солвера, ініціалізуючи всі необхідні параметри та \textbf{stability\_factor}, що обчислює значення у вузлі $(i\Delta x, j\Delta y)$ ф-ї $\psi$, яка виникає у виразі для коеф-та підсилення \eqref{eq:stable}.

Другий клас є своєрідним фреймворком для проведення усіх операцій (основних та підготовчих) по розв’язуванню задачі та відповідному зберіганню розв’язків; він жорстко зв’язується з моделлю, яку ми передаємо йому (\textbf{CNSolver.fit(CNModel)}). Для розв’язання кожної зі згенерованих моделей повинен бути створений свій солвер, у який "фітиться" модель і який агрегує відповідні розв’язки. Розв’язки у буфері солвера у процесі обробки зберігаються у так званій \textit{flat} моделі --- сітка, представлена у вигляді одновимірного numpy-масиву, рядок за рядком. Після обробки матричний вигляд відновлюється і записується у остаточний багатовимірний numpy-масив, кожен елемент якого представляє розв’язок на деякому кроці часу.

У наступному пункті ми розглянемо специфічні методи роботи з розрідженими масивами даних, які використовуються класом \textbf{CNSolver} для динамічної генерації  матриці системи у відповідному представленні та для оптимального розв’язання системи, представленої у подібному вигляді.\\

\subsection{Формати представлень розріджених масивів. Проекційні алгоритми}
Генерувати матрицю системи необхідно на кожній ітерації, оскільки її коеф-ти явно залежать від $\un{0}{0}$, саме тому цю допоміжну операцію було реалізовано всередині класу \textbf{CNSolver} --- відповідні значення додатково зберігаються у флет-моделі у буфері солвера і допомагають генерації. Як бачимо це вимагає значних витрат обчислювальних ресурсів, в тому числі й пам’яті машини, особливо якщо вимагати достатньо високої точності розв’язку. Матриця системи, як було вже вказано, є 5-діагональною і, за рахунок цього, дуже сильно розрідженою. Тому нами був використаний динамічний метод заповнення матриці виключно ненульовими ел-ми і подальше її конвертування в формат \textbf{DIAGONAL}. Сам формат описаний нижче. Динамічне заповнення забезпечив модуль \textbf{scipy.sparse}, у якому реалізований допоміжний формат \textbf{lil\_matrix}, який є своєрідним узагальненим шаблоном для будь-якого формату збереження розріджених матриць. Він дозволяє додавати ненульовий елмент з вказанням конкретної позиції і в подальшому трансформувати матрицю для використання у солвері.

У нашій реалізації ми використовували формат представлення \textbf{DIAGONAL}, на противагу іншому найбільш розповсюдженому формату \textbf{CSR} (Compressed Sparse Row), оскільки він безпосередньо був розроблений для n-діагональних розріджених матриць і конвертація у нього для таких матриць виконується дещо швидше. Детальний огляд конкретних форматів збереження можна знайти у \cite{saad}.

Збереження у форматі \textbf{DIAGONAL} передбачає наступне. Діагоналі зберігаються у прямокутному масиві \textbf{DIAG} розмірності $n\times Nd$, де Nd --- к-ть ненульових діагоналей. Кожень стовпець представляє діагональ і у діагоналей, що мають менше n елементів на відповідних позиціях стоять деякі фіктивні "заповнювачі". Зсуви діагоналей відносно головної мають бути відомі і записуються вони у окремий одновимірний масив \textbf{IOFF} довжини Nd. Значення 0 відповідає головній діагоналі, -1 --- першій субдіагоналі у нижньому трикутнику, +1 --- першій супердіагоналі у верхньому трикутнику і т.д. Послідовність вектор-стовпчиків, що представляють діагоналі та їх зсувів у масиві \textbf{IOFF} повинні співпадати, при цьому абсолютне розташування самих вектор-стовпчиків відносно один одного у \textbf{DIAG} зазвичай суттєвого значення не має, хоча в деяких випадках може бути корисним прямий порядок.

Система р-нь, що представлена розрідженою матрицею $LHS$ та вектор-стовпчиком $RHS$, що генеруються солвером на кожному кроці, представлена у специфічному форматі і повинна розв’язуватись за допомогою відповідних алгоритмів. Модуль \textbf{scipy.sparse.linalg} пропонує для розріджених матриць серію різних алгоритмів, що мають свої переваги та недоліки і ефективність та застосовність яких часто сильно залежить від властивостей матриці лівої частини системи, тому перед нами постала проблема вибору найкращого із доступних методів. Нами було проведено невелике дослідження швидкості обробки кожним з алгоритмів системи великих розмірностей ($90601\times90601 \approx 13$Mb у серіалізованому вигляді). Для кожного алгоритму проводилась серія запусків для 5 матриць однакової розмірності, вказаної вище, згенерованих солвером на першому кроці по часу для моделей з різним набором параметрів і записаних у дамп-файл. Абсолютна точність наближення задавалась порядку $10^{-6}$, на противагу значенню за замовчуванням для даного модуля: $10^{-4}$; початкове наближення явно не задавалось, так що за замовчуванням алгоритм стартував зі значення $\vec{0}$. Час виконання був усереднений для кожного з алгоритмів і наведений у таблиці нижче.
\vspace{10pt}
\begin{center}
	\begin{tabular}{|l|c|}
	\hline
	Algorithm & Time of estimation, sec \\ \hline
	GMRES & $1.77282$ \\ \hline
	QMR & $1.83367$ \\ \hline
	BiCGSTAB & $0.97049$\\ \hline
	LGMRES & $1.72245$\\ \hline
	\end{tabular}
\end{center}
\vspace{10pt}
Неважко, бачити, что суттево швидшим ніж інші є BiCGSTAB (Biconjugate Gradient Stabilised Algorithm), отже в якості солвера для нашої системи був обраний саме він.

Зробимо невелике зауваження щодо класу пропонованих алгоритмів. Всі вони детально розглянуті у \cite{saad}. Клас даних алгоритмів називається \textit{проекційними алгоритмами на підпростір Крилова}, або просто \textit{алгоритмами криловського типу}. Усі проекційні алгоритми спираються на ключову ідею, що чергове наближення для розв’язку лінійної системи $\bf A x = \bf b$ може бути шукане в деякому підпросторі $\mathbb{K}_m$ розмірності $m$ основного простору $\mathbb{R}^n$, який називається \textit{subspace of candidate approximants} і для забезпечення цього має бути накладено $m$ умов, що де-факто є умовами ортогональності шуканого наближення (якщо говорити більш точно, то чергової нев’язки, а не власне самого наближення) до $m$ базисних векторів іншого підпростору $\mathbb{L}_m$, який називають \textit{subspace of constraints}. Усі алгоритми криловського типу мають важливу спільну особливість: в якості $\mathbb{K}_m$ на кожному кроці $m$ обирається підпростір Крилова $\mathbb{K}_m(\textbf{A},\vec{v}) = span(\vec{v}, \textbf{А}\vec{v}, \dots, \textbf{А}^{m-1}\vec{v})$ для деякого вектора $\vec{v}$, який найчастіше покладають рівним $\vec{r}_0$ --- початковій нев’язці; тут \textbf{А} --- матриця системи, що розв’язується. Уся різноманітність алгоритмів криловського типу полягає у різних методах генерації базису чергової лінійної оболонки та відповідних апріорних припущеннях щодо вигляду матриці $\textbf{А}$. Чудовий опис алгоритму побудови біортогональних систем базисів для двох криловських підпросторів, один з яких спирається на саму матрицю $\textbf{А}$, а інший на $\textbf{А}^T$, і який є ключовим для обраного нами алгоритму  BiCGSTAB, можна знайти у оригінальній роботі автора даного алгоритму \cite{lanczos}.