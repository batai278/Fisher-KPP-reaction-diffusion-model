\section{Аналіз отриманих результатів}
Запуски програмної реалізації проводились здебільшого на сітках $101\times 101$ та $201\times 201$ для квадратних областей $[0,1]\times [0,1]$ та $[0,10]\times [0,10]$ з однаковими просторовими кроками $0,01$ та $0,1$ відповідно. Моделювання проводилось для 50 кроків по часу. Для $\lambda = 0,5, \lambda = 0,75$ обирався крок по часу $\Delta t = 0,01$, для $\lambda = 0,25$ (0 --- явна схема) заради забезпечення стійкості крок по часу обирався меншим, а саме у випадку $\Delta x = \Delta y = 0,01$ покладали $\Delta t = 0,0001$ і проводили моделювання для 100 кроків по часу. В якості $U(\vec{x})$ --- функції початкових умов була обрана гаусіана
\begin{equation}
2\left[exp\left\{-\frac{(x-0,5)^2}{2}\right\} + exp\left\{-\frac{(y-0,5)^2}{2}\right\}\right]
\end{equation}
для квадрату $[0,1]^2$ та
\begin{equation}
3\left[exp\left\{-\frac{(x-3)^2}{2\times 1,5^2}\right\} + exp\left\{-\frac{(y-2,5)^2}{2\times 1,5^2}\right\}\right]
\end{equation}
для квадрату $[0,10]^2$.
Графіки для цих двох випадків, представлені у формі heat-plot, наведено у \textbf{Додатку ІІ}.

Крім того, було проаналізовано порівняльну поведінку точності розв’язків при однакових кроках дискретизації абсолютно явної схеми без додаткової лінеаризації, що її описано у підрозділі 2.3, та пропонованої неявної схеми при $\lambda = 0.5\ (0,75)$. З апріорних оцінок відомо, що неявна схема має по часу вищий порядок точності, ніж явна, проте виникло питання, наскільки швидко у явній схемі накопичуються похибки. Проводилось моделювання при $\Delta t = 0,001\ \Delta x = \Delta y = 0,1\ $ на $[0,10]^2$ для 5 кроків по часу. Порівняння проводилось шляхом оцінки норми Фробеніуса різниці матриць, що представляють розв’язок, для кожного кроку по часу. Задавалась точність, з якою ці матриці повинні співпадати. У таблиці нижче наведено результати порівнянь. 
\begin{center}
	\begin{tabular}{|l|l|}
	\hline
	$\epsilon = 0.001$ & Is equal? \\ \hline
	1 & True \\ \hline
	2 & True \\ \hline
	3 & False \\ \hline
	4 & False \\ \hline
	5 & False \\ \hline
	\end{tabular}
\end{center} 
\vspace{10pt}
Як бачимо, вже після другого кроку по часу результати починають відрізнятись більше ніж на $0,001 = \Delta t$, тобто накопичення похибок у явній схемі відбувається катастрофічно швидко.

Ще одним суттєвим результатом є той факт, що в нелінійних моделях умова стійкості фон Неймана є лише необхідною, але не достатньою, внаслідок залежності коеф-ту підсилення від отриманого на поточному кроці результату розв’язання. На графіках нижче наведено результат моделювання для 5 кроків по часу на $[0,10]^2$ при тій же початковій ф-ї, що вказана вище і при таких значеннях параметрів моделі: $\alpha = -10,\ \beta = -1,\ m_0 = 2,\ \sigma = -0,01,\ \lambda = 0,75, \Delta t = 0,01, \Delta x = \Delta y = 0,1$. При цій комбінації параметрів умова фон Неймана для \eqref{eq:stable} виконується для всіх 5-ти кроків, при цьому на границі, де ф-я початкових умов, як і функція розв’язку швидко змінює характер росту і стрімко спадає, ми можемо бачити прояви нестійкості коливного харакеру.
\vspace{10pt}
\addtwoimghere{nonst2}{nonst3}{1}
\vspace{10pt}
\addtwoimghere{nonst4}{nonst5}{1}